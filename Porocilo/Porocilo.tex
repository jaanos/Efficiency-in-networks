\documentclass[a4paper, 16pt]{article}
\usepackage[slovene]{babel}
\usepackage[utf8]{inputenc}
\usepackage[T1]{fontenc}
\usepackage{lmodern}
\usepackage{multirow}
\usepackage{graphicx}
\usepackage{amsmath}
\usepackage{amssymb}
\usepackage{amsfonts}


\title{%
    Učinkovitost omrežij \\ 
    \large Poročilo}
\date{2020\\ November}
\author{Jure Babnik \\  Zala Stopar Špringer}

\begin{document}

\maketitle
\pagenumbering{gobble}


\newpage

\tableofcontents

\newpage
\pagenumbering{arabic}

\section{Priprava okolja}

Pred začetkom simulacij sva si pripravila delovno okolje. Za programerski del naloge sva uporabila \emph{Python}
in knjižnico \emph{Graph-Theory}.\\ 
Defirirala sva si funkcije, ki so nama ustvarile različne enostavne grafe, kot so mreže, 3-dimenzionalne mreže, 
popolna binomska drevesa, cikle, itd. Vsi grafi so neusmerjeni. Nato pa sva si še definirala funkcijo \emph{generate\_random}, 
ki sprejme število vozlišč $n$, na katerih funkcija naredi naključen usmerjen graf.\\
Prav tako sva si napisala funkcije, ki izračunajo učinkovitost omrežja. \\
\\
Formula za \textbf{povprečno učinkovitost} grafa $G$ je definirana kot:
$$ E(G) = \frac{1}{n(n-1)} \sum_{i\neq j \in G} \frac{1}{d(i,j)},$$

kjer je $d(i,j)$ dolžina najkrajše poti med $i$-to in $j$-to točko, $n$ pa je število vseh točk v grafu.\\
\\
\textbf{Globalna učinkovitost} je definirana kot:
$$ E_{glob}(G) = \frac{E(G)}{E(K_n)}, $$

kjer $K_n$, predstavlja poln graf na $n$ točkah.\\
\\
\textbf{Lokalna učinkovitost} je definirana kot:
$$ E_{loc}(G) = \frac{1}{n} \sum_{i \in G} E(G_i), $$

kjer $G_i$ predstavlja podgraf grafa $G$, ki je sestavljen le iz sosedov točke $i$ (brez točke $i$). \\

Vsa koda je zbrana v datoteki \emph{graphs.py}

\newpage

\section{Učinkovitost v preprostih grafih}



    \subsection{Mreže $m \times n$}
    Za nekaj različnih $m$ in $n$ sva ustvarila grafe in opazovala, kakšna je njihova 
    povprečna učinkovitost. 
    Rezultati so prikazani v spodnjih tabelah.

    \begin{table}[!h]
        \begin{tabular}{c|c|c|c|c|c|c|c}
            \multirow{2}{*}{$n$} & 
            \multicolumn{7}{c}{Povprečna učinkovitost}\\
               & $m=1$       & $m=2$           & $m=3$           & $m=4$         & $m = 5$     & $m = 10$    & $m=20$     \\ \hline
            2  & $1$         & $0.83333333$ & $0.7111111$ & $0.625$       & $0.5607404$ & $0.3842982$ & $0.2500188$\\
            3  & $0.8333333$ & $0.7111111$  & $0.6157407$ & $0.5464646$   & $0.4938095$ & $0.3453350$ & $0.2285143$\\
            4  & $0.7222222$ & $0.625$      & $0.5464646$ & $0.4883333$   & $0.4436090$ & $0.3150960$ & $0.2114040$\\
            5  & $0.6416667$ & $0.5607407$  & $0.4938095$ & $0.4436090$   & $0.4046429$ & $0.2910526$ & $0.1975453$\\
            10 & $0.4286596$ & $0.3842982$  & $0.3453350$ & $0.3150960$   & $0.2910526$ & $0.2176605$ & $0.1539108$\\
            20 & $0.2734463$ & $0.2500188$  & $0.2285143$ & $0.2114040$   & $0.1975453$ & $0.1539108$ & $0.1133842$\\

        \end{tabular}
        \caption{Povprečna učinkovitost $m \times n$ omrežij}
        \label{table: 1}
    \end{table}


    \subsection{3-dimenzionalne mreže}
    \begin{table}[!h]
        \begin{tabular}{c|c|c|c|c||c|c|c|c}
            \multirow{2}{*}{$n$} & 
            \multicolumn{4}{c}{$r = 2$} & \multicolumn{4}{c}{$r = 3$}\\
               & $m=2$       & $m=3$       & $m=5$       & $m=10$      & $m = 2$     & $m = 3$    & $m=5$        & $m=10$ \\ \hline
            2  & $0.6904761$ & $0.5959596$ & $0.4782456$ & $0.3357206$ & $0.5959596$ & $0.5193800$ & $0.4223864$ & $0.3018993$\\
            3  & $0.5959596$ & $0.5193800$ & $0.4223864$ & $0.3018993$ & $0.5193800$ & $0.4562206$ & $0.3753992$ & $0.2728478$\\
            5  & $0.4782456$ & $0.4223864$ & $0.3498866$ & $0.2565895$ & $0.4223864$ & $0.3753992$ & $0.3141630$ & $0.2339359$ \\
            10 & $0.3357206$ & $0.3018993$ & $0.2565895$ & $0.1954146$ & $0.3018993$ & $0.2728478$ & $0.2339359$ & $0.1805703$\\

        \end{tabular}
        \caption{Povprečna učinkovitost $m \times n \times r$ omrežij}
        \label{table: 2}
    \end{table}

    \begin{table}[!h]
        \begin{tabular}{c|c|c|c|c||c|c|c|c}
            \multirow{2}{*}{$n$} & 
            \multicolumn{4}{c}{$r = 5$} & \multicolumn{4}{c}{$r = 10$}\\
               & $m=2$       & $m=3$       & $m=5$       & $m=10$      & $m = 2$     & $m = 3$     & $m=5$       & $m=10$ \\ \hline
            2  & $0.4782456$ & $0.4223864$ & $0.3498866$ & $0.2565895$ & $0.3357206$ & $0.3018993$ & $0.2565895$ & $0.1954146$\\
            3  & $0.4223864$ & $0.3753992$ & $0.3141630$ & $0.2339359$ & $0.3018993$ & $0.2728478$ & $0.2339359$ & $0.1805793$\\
            5  & $0.3498866$ & $0.3141630$ & $0.2669560$ & $0.2032711$ & $0.2565895$ & $0.2339359$ & $0.2032711$ & $0.1600518$\\
            10 & $0.2565895$ & $0.2339359$ & $0.2032711$ & $0.1600518$ & $0.1954146$ & $0.1805703$ & $0.1600518$ & $0.1298527$\\

        \end{tabular}
        \caption{Povprečna učinkovitost $m \times n \times r$ omrežij}
        \label{table: 3}
    \end{table}

    \subsection{Cikli}
    \begin{table}[!h]
        \begin{tabular}{c|c}
            n & Povprečna učinkovitost \\ \hline
            3   & $1$ \\
            4   & $0.8333333$ \\
            5   & $0.75$ \\
            10  & $0.4851852$ \\
            20  & $0.3030493$ \\
            50  & $0.1549371$ \\
            100 & $0.0906910$ \\

        \end{tabular}
        \caption{Povprečna učinkovitost ciklov z $n$ točkami}
        \label{table: 4}
    \end{table}

    \subsection{Binomska drevesa}
    \begin{table}[!h]
        \begin{tabular}{c|c}
            n & Povprečna učinkovitost \\ \hline
            2  & $0.8333333$ \\
            3  & $0.5634921$ \\
            4  & $0.3907937$ \\
            5  & $0.2776549$ \\
            6  & $0.2028584$ \\
            7  & $0.1530067$ \\
            8  & $0.1193636$ \\
            9  & $0.0962377$ 

        \end{tabular}
        \caption{Povprečna učinkovitost popolnih binomskih dreves globine $n$}
        \label{table: 5}
    \end{table}

\section{Učinkovitost v naključnih grafih}

\section{Sklep}






\end{document}