\documentclass[a4paper, 16pt]{article}
\usepackage[slovene]{babel}
\usepackage[utf8]{inputenc}
\usepackage[T1]{fontenc}
\usepackage{lmodern}
\usepackage{multirow}
\usepackage{graphicx}
\usepackage{amsmath}
\usepackage{amssymb}
\usepackage{amsfonts}


\title{%
    Učinkovitost omrežij \\ 
    \large Poročilo}
\date{2020\\ November}
\author{Jure Babnik \\  Zala Stopar Špringer}

\begin{document}

\maketitle
\pagenumbering{gobble}


\newpage

\tableofcontents

\newpage
\pagenumbering{arabic}

\section{Priprava okolja}

Pred začetkom simulacij sva si pripravila delovno okolje. Za programerski del naloge sva uporabila \emph{Python}
in knjižnico \emph{Graph-Theory}.\\ 
Defirirala sva si funkcije, ki so nama ustvarile različne enostavne grafe, kot so mreže, 3-dimenzionalne mreže, 
popolna binomska drevesa, cikle, itd.
Prav tako sva si napisala funkcije, ki izračunajo učinkovitost omrežja. \\
\\
Formula za \textbf{povprečno učinkovitost} grafa $G$ je definirana kot:
$$ E(G) = \frac{1}{n(n-1)} \sum_{i\neq j \in G} \frac{1}{d(i,j)},$$

kjer je $d(i,j)$ dolžina najkrajše poti med $i$-to in $j$-to točko, $n$ pa je število vseh točk v grafu.\\
\\
\textbf{Globalna učinkovitost} je definirana kot:
$$ E_{glob}(G) = \frac{E(G)}{E(K_n)}, $$

kjer $K_n$, predstavlja poln graf na $n$ točkah.\\
\\
\textbf{Lokalna učinkovitost} je definirana kot:
$$ E_{loc}(G) = \frac{1}{n} \sum_{i \in G} E(G_i), $$

kjer $G_i$ predstavlja podgraf grafa $G$, ki je sestavljen le iz sosedov točke $i$ (brez točke $i$). \\
\\
Vsa koda je zbrana v datoteki \emph{graphs.py}


\section{Učinkovitost v preprostih grafih}



    \subsection{Mreže $m \times n$}
    Za nekaj različnih $m$ in $n$ sva ustvarila grafe in opazovala, kakšno učinkovitost imajo.
    Rezultati so prikazani v spodnjih tabelah.

    \begin{table}[h!]
        \begin{tabular}{c|c|c|c}
            \multirow{2}{*}{$n$} & 
            \multicolumn{3}{c}{$m = 1$}\\
            & Povprečna učinkovitost & Globalna učinkovitost & Lokalna učinkovitost \\ \hline
            2 & 1 & 1 & 0 \\
            3 & $\frac{5}{6}$ & $\frac{5}{6}$ & 0 \\
            4 & $\frac{13}{18}$ & $\frac{13}{18}$ & 0 \\
            5 & $\frac{77}{120}$ & $\frac{77}{120}$ & 0 \\
            10 & 0.4286596 & 0.4286596 & 0\\
            20 & 0.2734463 & 0.2734463 & 0\\

        \end{tabular}
        \caption{Učinkovitost $m \times n$ omrežij, $m = 1$}
        \label{table: 1}
    \end{table}


    \subsection{3-dimenzionalne mreže}

    \subsection{Cikli}

    \subsection{Binomska drevesa}

\section{Učinkovitost v naključnih grafih}

\section{Sklep}






\end{document}